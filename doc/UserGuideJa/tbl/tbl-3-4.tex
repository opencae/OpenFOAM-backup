%#! platex UserGuideJa
\begin{tabularx}{\textwidth}{lXp{10zw}}
 \multicolumn{3}{l}{必須入力} \\
 \hline
\index{numberOfSubdomains@\OFkeyword{numberOfSubdomains}!キーワード}%
\index{キーワード!numberOfSubdomains@\OFkeyword{numberOfSubdomains}}%
 \OFkeyword{numberOfSubdomains} & サブドメインの総数 & $N$ \\
\index{method@\OFkeyword{method}!キーワード}%
\index{キーワード!method@\OFkeyword{method}}%
 \OFkeyword{method} & 分割方法 &
\index{simple@\OFkeyword{simple}!キーワードエントリ}%
\index{キーワードエントリ!simple@\OFkeyword{simple}}%
         \OFkeyword{simple}/\hfil\break
\index{hierarchical@\OFkeyword{hierarchical}!キーワードエントリ}%
\index{キーワードエントリ!hierarchical@\OFkeyword{hierarchical}}%
         \OFkeyword{hierarchical}/\hfil\break
         \OFkeyword{scotch}/
\index{metis@\OFkeyword{metis}!キーワードエントリ}%
\index{キーワードエントリ!metis@\OFkeyword{metis}}%
         \OFkeyword{metis}/
\index{manual/@\OFkeyword{manual/}!キーワードエントリ}%
\index{キーワードエントリ!manual/@\OFkeyword{manual/}}%
         \OFkeyword{manual/} \\
 \\
 \multicolumn{3}{l}{\OFkeyword{simpleCoeffs}エントリ} \\
 \hline
\index{n@\OFkeyword{n}!キーワード}%
\index{キーワード!n@\OFkeyword{n}}%
 \OFkeyword{n} & $x$,$y$,$z$のサブドメイン数 & $(n_{x}, n_{y}, n_{z})$ \\
\index{delta@\OFkeyword{delta}!キーワード}%
\index{キーワード!delta@\OFkeyword{delta}}%
 \OFkeyword{delta} & セルのスキュー因数 & 一般的には,$10^{-3}$ \\
 \\
 \multicolumn{3}{l}{\OFkeyword{hierarchicalCoeffs}エントリ} \\
 \hline
 \OFkeyword{n} & $x$,$y$,$z$のサブドメイン数 & $(n_{x}, n_{y}, n_{z})$ \\
 \OFkeyword{delta} & セルのスキュー因数 & 一般的には,$10^{-3}$ \\
\index{order@\OFkeyword{order}!キーワード}%
\index{キーワード!order@\OFkeyword{order}}%
 \OFkeyword{order} & 分割の順序 & \OFkeyword{xyz}/\OFkeyword{xzy}/\OFkeyword{yzx}... \\
 \\
 \multicolumn{3}{l}{%
\index{scotchCoeffs@\OFkeyword{scotchCoeffs}!キーワード}%
\index{キーワード!scotchCoeffs@\OFkeyword{scotchCoeffs}}%
 \OFkeyword{scotchCoeffs}エントリ} \\
 \hline
\index{processorWeights@\OFkeyword{processorWeights}!キーワード}%
\index{キーワード!processorWeights@\OFkeyword{processorWeights}}%
 \OFkeyword{processorWeights}
 & プロセッサへのセルの割当の重み係数の一覧.
     例:\verb|<wt1>|はプロセッサ1の重み係数.
     重みは規格化され,
     どんな範囲の値も取ることが可能.
     & (\verb|<wt1>|...\verb|<wtN>|) \\
 \OFkeyword{strategy}
 & 分割の戦略.デフォルトは\OFkeyword{"b"}
     &  \\
 \\
 \multicolumn{3}{l}{%
\index{metisCoeffs@\OFkeyword{metisCoeffs}!キーワード}%
\index{キーワード!metisCoeffs@\OFkeyword{metisCoeffs}}%
 \OFkeyword{metisCoeffs}エントリ} \\
 \hline
\index{processorWeights@\OFkeyword{processorWeights}!キーワード}%
\index{キーワード!processorWeights@\OFkeyword{processorWeights}}%
 \OFkeyword{processorWeights}
 & 同上
     & (\verb|<wt1>|...\verb|<wtN>|) \\
 \\
 \multicolumn{3}{l}{%
\index{manualCoeffs@\OFkeyword{manualCoeffs}!キーワード}%
\index{キーワード!manualCoeffs@\OFkeyword{manualCoeffs}}%
 \OFkeyword{manualCoeffs}エントリ} \\
 \hline
 \OFkeyword{dataFile} & プロセッサへのセルの割当のデータを含むファイル名 & \texttt{"<fileName>"} \\
 \\
 \multicolumn{3}{l}{分散型データの入力(オプション)---\autoref{ssec:3.4.3}参照} \\
 \hline
\index{distributed@\OFkeyword{distributed}!キーワード}%
\index{キーワード!distributed@\OFkeyword{distributed}}%
 \OFkeyword{distributed} & データはいくつかのディスクのに分散しますか? & \OFkeyword{yes}/\OFkeyword{no} \\
\index{roots@\OFkeyword{roots}!キーワード}%
\index{キーワード!roots@\OFkeyword{roots}}%
 \OFkeyword{roots} & ケースディレクトリへのルートパス.
     例:\verb|<rt1>|はノード1へのルートパス
     & (\verb|<rt1>|...\verb|<rtN>|) \\
\end{tabularx}
