%#! platex UserGuideJa
\begin{longtable}{lX}
 \multicolumn{2}{l}{基本熱物理モデル ---
\index{basicThermophysicalModels@\string\OFclass{basicThermophysicalModels}!ライブラリ}%
\index{ライブラリ!basicThermophysicalModels@\string\OFclass{basicThermophysicalModels}}%
 \OFclass{basicThermophysicalModels}} \\
 \hline
\index{hPsiThermo@\OFclass{hPsiThermo}!モデル}%
\index{モデル!hPsiThermo@\OFclass{hPsiThermo}}%
 \OFclass{hPsiThermo} &
     エンタルピ$h$と圧縮率$\psi$に基づく一般熱物理モデル計算 \\
\index{hsPsiThermo@\OFclass{hsPsiThermo}!モデル}%
\index{モデル!hsPsiThermo@\OFclass{hsPsiThermo}}%
 \OFclass{hsPsiThermo} &
\OFrevision*{sensible enthalpy?}%
     顕在エンタルピ$h_{\mathrm{s}}$と圧縮率$\psi$に基づく一般熱物理モデル計算 \\
\index{ePsiThermo@\OFclass{ePsiThermo}!モデル}%
\index{モデル!ePsiThermo@\OFclass{ePsiThermo}}%
 \OFclass{ePsiThermo} &
     内部エネルギ$e$と圧縮率$\psi$に基づく一般熱物理モデル計算 \\
\index{hRhoThermo@\OFclass{hRhoThermo}!モデル}%
\index{モデル!hRhoThermo@\OFclass{hRhoThermo}}%
 \OFclass{hRhoThermo} &
     エンタルピ$h$に基づく一般熱物理モデル計算 \\
\index{hsRhoThermo@\OFclass{hsRhoThermo}!モデル}%
\index{モデル!hsRhoThermo@\OFclass{hsRhoThermo}}%
 \OFclass{hsRhoThermo}%
\footnote{訳注:原文では\OFclass{hRhoThermo}となっているが,おそらく誤植.}%
 &
     顕在エンタルピ$h_{\mathrm{s}}$に基づく一般熱物理モデル計算 \\
\index{pureMixture@\OFclass{pureMixture}!モデル}%
\index{モデル!pureMixture@\OFclass{pureMixture}}%
 \OFclass{pureMixture} &
     パッシブガス混合物の一般熱物理モデル計算 \\
 \\
 \multicolumn{2}{l}{化学反応モデル ---
\index{reactionThermophysicalModels@\string\OFclass{reactionThermophysicalModels}!ライブラリ}%
\index{ライブラリ!reactionThermophysicalModels@\string\OFclass{reactionThermophysicalModels}}%
 \OFclass{reactionThermophysicalModels}} \\
 \hline
\index{hPsiMixtureThermo@\OFclass{hPsiMixtureThermo}!モデル}%
\index{モデル!hPsiMixtureThermo@\OFclass{hPsiMixtureThermo}}%
 \OFclass{hPsiMixtureThermo} &
 エンタルピ$h$と$\psi$に基づいて混合気燃焼のエンタルピを計算する \\
\index{hsPsiMixtureThermo@\OFclass{hsPsiMixtureThermo}!モデル}%
\index{モデル!hsPsiMixtureThermo@\OFclass{hsPsiMixtureThermo}}%
 \OFclass{hsPsiMixtureThermo} &
 顕在エンタルピ$h_{\mathrm{s}}$と$\psi$に基づいて混合気燃焼のエンタルピを計算する \\
\index{hRhoMixtureThermo@\OFclass{hRhoMixtureThermo}!モデル}%
\index{モデル!hRhoMixtureThermo@\OFclass{hRhoMixtureThermo}}%
 \OFclass{hRhoMixtureThermo} &
 エンタルピ$h$と$\rho$に基づいて混合気燃焼のエンタルピを計算する \\
\index{hsRhoMixtureThermo@\OFclass{hsRhoMixtureThermo}!モデル}%
\index{モデル!hsRhoMixtureThermo@\OFclass{hsRhoMixtureThermo}}%
 \OFclass{hsRhoMixtureThermo} &
 顕在エンタルピ$h_{\mathrm{s}}$と$\rho$に基づいて混合気燃焼のエンタルピを計算する \\
\index{hhuMixtureThermo@\OFclass{hhuMixtureThermo}!モデル}%
\index{モデル!hhuMixtureThermo@\OFclass{hhuMixtureThermo}}%
 \OFclass{hhuMixtureThermo} &
     不燃気体と混合気のエンタルピ計算 \\
\index{homogeneousMixture@\OFclass{homogeneousMixture}!モデル}%
\index{モデル!homogeneousMixture@\OFclass{homogeneousMixture}}%
 \OFclass{homogeneousMixture} &
     標準燃料質量分率$b$に基づく混合気燃焼 \\
\index{inhomogeneousMixture@\OFclass{inhomogeneousMixture}!モデル}%
\index{モデル!inhomogeneousMixture@\OFclass{inhomogeneousMixture}}%
 \OFclass{inhomogeneousMixture} &
     $b$と総燃料質量分率$f_{\mathrm{t}}$に基づく混合気燃焼 \\
\index{veryInhomogeneousMixture@\OFclass{veryInhomogeneousMixture}!モデル}%
\index{モデル!veryInhomogeneousMixture@\OFclass{veryInhomogeneousMixture}}%
 \OFclass{veryInhomogeneousMixture} &
     $b$,$f_{\mathrm{t}}$と
     不燃燃料質量分率$f_{\mathrm{u}}$に基づく混合気燃焼 \\
\index{dieselMixture@\OFclass{dieselMixture}!モデル}%
\index{モデル!dieselMixture@\OFclass{dieselMixture}}%
 \OFclass{dieselMixture} &
     $f_{\mathrm{t}}$と$f_{\mathrm{u}}$に基づく混合気燃焼 \\
\index{basicMultiComponentMixture@\OFclass{basicMultiComponentMixture}!モデル}%
\index{モデル!basicMultiComponentMixture@\OFclass{basicMultiComponentMixture}}%
 \OFclass{basicMultiComponentMixture} &
     複数の要素に基づく基本的な混合気 \\
\index{multiComponentMixture@\OFclass{multiComponentMixture}!モデル}%
\index{モデル!multiComponentMixture@\OFclass{multiComponentMixture}}%
 \OFclass{multiComponentMixture} &
     複数の要素に基づく派生混合気 \\
\index{reactingMixture@\OFclass{reactingMixture}!モデル}%
\index{モデル!reactingMixture@\OFclass{reactingMixture}}%
 \OFclass{reactingMixture} &
     熱力学と反応スキームによる燃焼混合気 \\
\index{egrMixture@\OFclass{egrMixture}!モデル}%
\index{モデル!egrMixture@\OFclass{egrMixture}}%
 \OFclass{egrMixture} &
     排気再循環の混合気 \\
 \\
 \multicolumn{2}{l}{輻射モデル ---
\index{radiationModels@\string\OFclass{radiationModels}!ライブラリ}%
\index{ライブラリ!radiationModels@\string\OFclass{radiationModels}}%
 \OFclass{radiationModels}} \\
 \hline
\index{P1@\OFclass{P1}!ライブラリ}%
\index{ライブラリ!P1@\OFclass{P1}}%
 \OFclass{P1} &
     P1モデル \\
\index{fvDOM@\OFclass{fvDOM}!ライブラリ}%
\index{ライブラリ!fvDOM@\OFclass{fvDOM}}%
 \OFclass{fvDOM} &
     有限体積離散座標法 \\
\index{viewFactor@\OFclass{viewFactor}!ライブラリ}%
\index{ライブラリ!viewFactor@\OFclass{viewFactor}}%
 \OFclass{viewFactor} &
     形態係数の輻射モデル \\
 \\
 \multicolumn{2}{l}{層流火炎速度モデル ---
\index{laminarFlameSpeedModels@\string\OFclass{laminarFlameSpeedModels}!ライブラリ}%
\index{ライブラリ!laminarFlameSpeedModels@\string\OFclass{laminarFlameSpeedModels}}%
 \OFclass{laminarFlameSpeedModels}} \\
 \hline
\index{constLaminarFlameSpeed@\OFclass{constLaminarFlameSpeed}!モデル}%
\index{モデル!constLaminarFlameSpeed@\OFclass{constLaminarFlameSpeed}}%
 \OFclass{constLaminarFlameSpeed} &
     一定層流火炎速度 \\
\index{GuldersLaminarFlameSpeed@\OFclass{GuldersLaminarFlameSpeed}!モデル}%
\index{モデル!GuldersLaminarFlameSpeed@\OFclass{GuldersLaminarFlameSpeed}}%
 \OFclass{GuldersLaminarFlameSpeed} &
     Gulderの層流火炎速度モデル \\
\index{GuldersEGRLaminarFlameSpeed@\OFclass{GuldersEGRLaminarFlameSpeed}!モデル}%
\index{モデル!GuldersEGRLaminarFlameSpeed@\OFclass{GuldersEGRLaminarFlameSpeed}}%
 \OFclass{GuldersEGRLaminarFlameSpeed} &
     排気再循環モデルを伴うGulderの層流火炎速度モデル \\
 \\
 \multicolumn{2}{l}{バロトロピック圧縮性モデル ---
\index{barotropicCompressibilityModels@\string\OFclass{barotropicCompressibilityModels}!ライブラリ}%
\index{ライブラリ!barotropicCompressibilityModels@\string\OFclass{barotropicCompressibilityModels}}%
 \OFclass{barotropicCompressibilityModels}} \\
 \hline
\index{linear@\OFclass{linear}!ライブラリ}%
\index{ライブラリ!linear@\OFclass{linear}}%
 \OFclass{linear} &
     線形圧縮性モデル \\
\index{Chung@\OFclass{Chung}!ライブラリ}%
\index{ライブラリ!Chung@\OFclass{Chung}}%
 \OFclass{Chung} &
     Chungの圧縮性モデル \\
\index{Wallis@\OFclass{Wallis}!ライブラリ}%
\index{ライブラリ!Wallis@\OFclass{Wallis}}%
 \OFclass{Wallis} &
     Wallisの圧縮性モデル \\
 \\
 \\
 \multicolumn{2}{l}{ガス種の熱物理特性 ---
\index{specie@\string\OFclass{specie}!ライブラリ}%
\index{ライブラリ!specie@\string\OFclass{specie}}%
 \OFclass{specie}} \\
 \hline
\index{icoPolynomial@\OFclass{icoPolynomial}!モデル}%
\index{モデル!icoPolynomial@\OFclass{icoPolynomial}}%
 \OFclass{icoPolynomial} &
     液体などの非圧縮性流体に対する多項式の状態方程式 \\
\index{perfectGas@\OFclass{perfectGas}!モデル}%
\index{モデル!perfectGas@\OFclass{perfectGas}}%
 \OFclass{perfectGas} &
     理想気体に対する状態方程式 \\
\index{eConstThermo@\OFclass{eConstThermo}!モデル}%
\index{モデル!eConstThermo@\OFclass{eConstThermo}}%
 \OFclass{eConstThermo} &
     内部エネルギ$e$とエントロピ$s$に関する一定比熱$c_{\mathrm{p}}$モデル \\
\index{hConstThermo@\OFclass{hConstThermo}!モデル}%
\index{モデル!hConstThermo@\OFclass{hConstThermo}}%
 \OFclass{hConstThermo} &
     エンタルピ$h$とエントロピ$s$に関する一定比熱$c_{\mathrm{p}}$モデル \\
\index{hPolynomialThermo@\OFclass{hPolynomialThermo}!モデル}%
\index{モデル!hPolynomialThermo@\OFclass{hPolynomialThermo}}%
 \OFclass{hPolynomialThermo} &
     $h$と$s$を評価する多項式の係数の関数により$c_{\mathrm{p}}$が評価される \\
\index{janafThermo@\OFclass{janafThermo}!モデル}%
\index{モデル!janafThermo@\OFclass{janafThermo}}%
 \OFclass{janafThermo} &
     $h$や$s$のようなJANAF熱力学テーブルの係数をもつ関数によって評価した$c_{\mathrm{p}}$ \\
\index{specieThermo@\OFclass{specieThermo}!モデル}%
\index{モデル!specieThermo@\OFclass{specieThermo}}%
 \OFclass{specieThermo} &
     $c_{\mathrm{p}}$,$h$そして/または$s$から派生するような熱物理特性 \\
\index{constTransport@\OFclass{constTransport}!モデル}%
\index{モデル!constTransport@\OFclass{constTransport}}%
 \OFclass{constTransport} &
     一定の輸送特性 \\
\index{polynomialTransport@\OFclass{polynomialTransport}!モデル}%
\index{モデル!polynomialTransport@\OFclass{polynomialTransport}}%
 \OFclass{polynomialTransport} &
     多項式に基づく温度依存輸送特性 \\
\index{sutherlandTransport@\OFclass{sutherlandTransport}!モデル}%
\index{モデル!sutherlandTransport@\OFclass{sutherlandTransport}}%
 \OFclass{sutherlandTransport} &
     温度依存輸送特性のためのサザーランドの公式 \\
 \\
 \multicolumn{2}{l}{熱物理特性の関数/表 ---
\index{thermophysicalFunctions@\string\OFclass{thermophysicalFunctions}!ライブラリ}%
\index{ライブラリ!thermophysicalFunctions@\string\OFclass{thermophysicalFunctions}}%
 \OFclass{thermophysicalFunctions}} \\
 \hline
\index{NSRDSfunctions@\OFclass{NSRDSfunctions}!モデル}%
\index{モデル!NSRDSfunctions@\OFclass{NSRDSfunctions}}%
 \OFclass{NSRDSfunctions} &
     標準参照データシステム (NSRDS) - 米国化学工学会 (AICHE) のデータ編集表 \\
\index{APIfunctions@\OFclass{APIfunctions}!モデル}%
\index{モデル!APIfunctions@\OFclass{APIfunctions}}%
 \OFclass{APIfunctions} &
     蒸気拡散のための米国石油協会 (API) の関数 \\
 \\
%  \multicolumn{2}{l}{確率密度関数 ---
% \index{pdf@\string\OFclass{pdf}!ライブラリ}%
% \index{ライブラリ!pdf@\string\OFclass{pdf}}%
%  \OFclass{pdf}} \\
%  \hline
% \index{RosinRammler@\OFclass{RosinRammler}!モデル}%
% \index{モデル!RosinRammler@\OFclass{RosinRammler}}%
%  \OFclass{RosinRammler} &
%      ロジン・ラムラー分布 \\
% \index{normal@\OFclass{normal}!モデル}%
% \index{モデル!normal@\OFclass{normal}}%
%  \OFclass{normal} &
%      正規分布 \\
% \index{uniform@\OFclass{uniform}!モデル}%
% \index{モデル!uniform@\OFclass{uniform}}%
%  \OFclass{uniform} &
%      一様分布 \\
% \index{exponential@\OFclass{exponential}!モデル}%
% \index{モデル!exponential@\OFclass{exponential}}%
%  \OFclass{exponential} &
%      指数分布 \\
% \index{general@\OFclass{general}!モデル}%
% \index{モデル!general@\OFclass{general}}%
%  \OFclass{general} &
%      一般化分布 \\
%  \\
 \multicolumn{2}{l}{化学モデル ---
\index{chemistryModel@\string\OFclass{chemistryModel}!ライブラリ}%
\index{ライブラリ!chemistryModel@\string\OFclass{chemistryModel}}%
 \OFclass{chemistryModel}} \\
\index{chemistryModel@\OFclass{chemistryModel}!モデル}%
\index{モデル!chemistryModel@\OFclass{chemistryModel}}%
 \OFclass{chemistryModel} &
     化学反応モデル \\
\index{chemistrySolver@\OFclass{chemistrySolver}!モデル}%
\index{モデル!chemistrySolver@\OFclass{chemistrySolver}}%
 \OFclass{chemistrySolver} &
     化学反応ソルバ
 \\
 \multicolumn{2}{l}{その他のライブラリ} \\
\index{liquidProperties@\OFclass{liquidProperties}!ライブラリ}%
\index{ライブラリ!liquidProperties@\OFclass{liquidProperties}}%
 \OFclass{liquidProperties} &
     液体の熱物性 \\
\index{liquidMixtureProperties@\OFclass{liquidMixtureProperties}!ライブラリ}%
\index{ライブラリ!liquidMixtureProperties@\OFclass{liquidMixtureProperties}}%
 \OFclass{liquidMixtureProperties} &
     混合液体の熱物性 \\
\index{basicSolidThermo@\OFclass{basicSolidThermo}!ライブラリ}%
\index{ライブラリ!basicSolidThermo@\OFclass{basicSolidThermo}}%
 \OFclass{basicSolidThermo} &
     固体の熱物理モデル \\
\index{solid@\OFclass{solid}!ライブラリ}%
\index{ライブラリ!solid@\OFclass{solid}}%
 \OFclass{solid} &
     固体の熱力学モデル \\
\index{SLGThermo@\OFclass{SLGThermo}!ライブラリ}%
\index{ライブラリ!SLGThermo@\OFclass{SLGThermo}}%
 \OFclass{SLGThermo} &
     固体・液体・気体の熱力学モデル \\
\index{solidProperties@\OFclass{solidProperties}!ライブラリ}%
\index{ライブラリ!solidProperties@\OFclass{solidProperties}}%
 \OFclass{solidProperties} &
     固体の熱物性 \\
\index{solidMixtureProperties@\OFclass{solidMixtureProperties}!ライブラリ}%
\index{ライブラリ!solidMixtureProperties@\OFclass{solidMixtureProperties}}%
 \OFclass{solidMixtureProperties} &
     混合固体の熱物性 \\
\index{thermalPorousZone@\OFclass{thermalPorousZone}!ライブラリ}%
\index{ライブラリ!thermalPorousZone@\OFclass{thermalPorousZone}}%
 \OFclass{thermalPorousZone} &
     エネルギ式の項を含んだセル領域に基づく多孔質領域の定義
\end{longtable}
