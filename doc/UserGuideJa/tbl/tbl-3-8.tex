%#! platex UserGuideJa
\begin{longtable}{lX}
 \multicolumn{2}{l}{基本熱物理モデル ---
\index{basicThermophysicalModels@\string\OFclass{basicThermophysicalModels}!ライブラリ}%
\index{ライブラリ!basicThermophysicalModels@\string\OFclass{basicThermophysicalModels}}%
 \OFclass{basicThermophysicalModels}} \\
 \hline
\index{hePsiThermo@\OFclass{hePsiThermo}!モデル}%
\index{モデル!hePsiThermo@\OFclass{hePsiThermo}}%
 \OFclass{hePsiThermo} &
     圧縮率$\psi$に基づく一般熱物理モデル計算 \\
\index{heRhoThermo@\OFclass{heRhoThermo}!モデル}%
\index{モデル!heRhoThermo@\OFclass{heRhoThermo}}%
 \OFclass{heRhoThermo} &
     密度$\rho$に基づく一般熱物理モデル計算 \\
\index{pureMixture@\OFclass{pureMixture}!モデル}%
\index{モデル!pureMixture@\OFclass{pureMixture}}%
 \OFclass{pureMixture} &
     不活性混合気体の一般熱物理モデル計算 \\
 \\
 \multicolumn{2}{l}{化学反応モデル ---
\index{reactionThermophysicalModels@\string\OFclass{reactionThermophysicalModels}!ライブラリ}%
\index{ライブラリ!reactionThermophysicalModels@\string\OFclass{reactionThermophysicalModels}}%
 \OFclass{reactionThermophysicalModels}} \\
 \hline
\index{psiReactionThermo@\OFclass{psiReactionThermo}!モデル}%
\index{モデル!psiReactionThermo@\OFclass{psiReactionThermo}}%
 \OFclass{psiReactionThermo} &
 $\psi$に基づいて燃焼混合気のエンタルピを計算する \\
\index{psiuReactionThermo@\OFclass{psiuReactionThermo}!モデル}%
\index{モデル!psiuReactionThermo@\OFclass{psiuReactionThermo}}%
 \OFclass{psiuReactionThermo} &
 $\psi_{\mathrm{u}}$に基づいて燃焼混合気のエンタルピを計算する \\
\index{rhoReactionThermo@\OFclass{rhoReactionThermo}!モデル}%
\index{モデル!rhoReactionThermo@\OFclass{rhoReactionThermo}}%
 \OFclass{rhoReactionThermo} &
 $\rho$に基づいて燃焼混合気のエンタルピを計算する \\
\index{heheupsiReactionThermo@\OFclass{heheupsiReactionThermo}!モデル}%
\index{モデル!heheupsiReactionThermo@\OFclass{heheupsiReactionThermo}}%
 \OFclass{heheupsiReactionThermo} &
 未燃ガスおよび燃焼混合気のエンタルピを計算する \\
\index{homogeneousMixture@\OFclass{homogeneousMixture}!モデル}%
\index{モデル!homogeneousMixture@\OFclass{homogeneousMixture}}%
 \OFclass{homogeneousMixture} &
     規格化燃料質量分率$b$に基づく混合気燃焼 \\
\index{inhomogeneousMixture@\OFclass{inhomogeneousMixture}!モデル}%
\index{モデル!inhomogeneousMixture@\OFclass{inhomogeneousMixture}}%
 \OFclass{inhomogeneousMixture} &
     $b$と総燃料質量分率$f_{\mathrm{t}}$に基づく混合気燃焼 \\
\index{veryInhomogeneousMixture@\OFclass{veryInhomogeneousMixture}!モデル}%
\index{モデル!veryInhomogeneousMixture@\OFclass{veryInhomogeneousMixture}}%
 \OFclass{veryInhomogeneousMixture} &
     $b$,$f_{\mathrm{t}}$と
     不燃燃料質量分率$f_{\mathrm{u}}$に基づく混合気燃焼 \\
\index{basicMultiComponentMixture@\OFclass{basicMultiComponentMixture}!モデル}%
\index{モデル!basicMultiComponentMixture@\OFclass{basicMultiComponentMixture}}%
 \OFclass{basicMultiComponentMixture} &
     複数の要素に基づく基本的な混合気 \\
\index{multiComponentMixture@\OFclass{multiComponentMixture}!モデル}%
\index{モデル!multiComponentMixture@\OFclass{multiComponentMixture}}%
 \OFclass{multiComponentMixture} &
     複数の要素に基づく派生混合気 \\
\index{reactingMixture@\OFclass{reactingMixture}!モデル}%
\index{モデル!reactingMixture@\OFclass{reactingMixture}}%
 \OFclass{reactingMixture} &
     熱力学と反応スキームによる燃焼混合気 \\
\index{egrMixture@\OFclass{egrMixture}!モデル}%
\index{モデル!egrMixture@\OFclass{egrMixture}}%
 \OFclass{egrMixture} &
     排気再循環の混合気 \\
\index{singleStepReactingMixture@\OFclass{singleStepReactingMixture}!モデル}%
\index{モデル!singleStepReactingMixture@\OFclass{singleStepReactingMixture}}%
 \OFclass{singleStepReactingMixture} &
     素反応を伴う混合気 \\
 \\
 \multicolumn{2}{l}{輻射モデル ---
\index{radiationModels@\string\OFclass{radiationModels}!ライブラリ}%
\index{ライブラリ!radiationModels@\string\OFclass{radiationModels}}%
 \OFclass{radiationModels}} \\
 \hline
\index{P1@\OFclass{P1}!ライブラリ}%
\index{ライブラリ!P1@\OFclass{P1}}%
 \OFclass{P1} &
     P1モデル \\
\index{fvDOM@\OFclass{fvDOM}!ライブラリ}%
\index{ライブラリ!fvDOM@\OFclass{fvDOM}}%
 \OFclass{fvDOM} &
     有限体積離散座標法 \\
\index{opaqueSolid@\OFclass{opaqueSolid}!ライブラリ}%
\index{ライブラリ!opaqueSolid@\OFclass{opaqueSolid}}%
 \OFclass{opaqueSolid} &
     不透明な固体の輻射.
     エネルギ式のソース項には何も影響しない(ゼロを返す)が,
     \OFclass{absorptionEmissionModel}および\OFclass{scatterModel}を作る. \\
\index{viewFactor@\OFclass{viewFactor}!ライブラリ}%
\index{ライブラリ!viewFactor@\OFclass{viewFactor}}%
 \OFclass{viewFactor} &
     形態係数の輻射モデル \\
 \\
 \multicolumn{2}{l}{層流火炎速度モデル ---
\index{laminarFlameSpeedModels@\string\OFclass{laminarFlameSpeedModels}!ライブラリ}%
\index{ライブラリ!laminarFlameSpeedModels@\string\OFclass{laminarFlameSpeedModels}}%
 \OFclass{laminarFlameSpeedModels}} \\
 \hline
\index{constLaminarFlameSpeed@\OFclass{constLaminarFlameSpeed}!モデル}%
\index{モデル!constLaminarFlameSpeed@\OFclass{constLaminarFlameSpeed}}%
 \OFclass{constLaminarFlameSpeed} &
     一定の層流火炎速度 \\
\index{GuldersLaminarFlameSpeed@\OFclass{GuldersLaminarFlameSpeed}!モデル}%
\index{モデル!GuldersLaminarFlameSpeed@\OFclass{GuldersLaminarFlameSpeed}}%
 \OFclass{GuldersLaminarFlameSpeed} &
     Gulderの層流火炎速度モデル \\
\index{GuldersEGRLaminarFlameSpeed@\OFclass{GuldersEGRLaminarFlameSpeed}!モデル}%
\index{モデル!GuldersEGRLaminarFlameSpeed@\OFclass{GuldersEGRLaminarFlameSpeed}}%
 \OFclass{GuldersEGRLaminarFlameSpeed} &
     排気再循環モデルを伴うGulderの層流火炎速度モデル \\
\index{RaviPetersen@\OFclass{RaviPetersen}!モデル}%
\index{モデル!RaviPetersen@\OFclass{RaviPetersen}}%
 \OFclass{RaviPetersen} &
     RaviとPetersenの相互関係から層流火炎速度を得る. \\
 \\
 \multicolumn{2}{l}{バロトロピック圧縮性モデル ---
\index{barotropicCompressibilityModels@\string\OFclass{barotropicCompressibilityModels}!ライブラリ}%
\index{ライブラリ!barotropicCompressibilityModels@\string\OFclass{barotropicCompressibilityModels}}%
 \OFclass{barotropicCompressibilityModels}} \\
 \hline
\index{linear@\OFclass{linear}!ライブラリ}%
\index{ライブラリ!linear@\OFclass{linear}}%
 \OFclass{linear} &
     線形圧縮性モデル \\
\index{Chung@\OFclass{Chung}!ライブラリ}%
\index{ライブラリ!Chung@\OFclass{Chung}}%
 \OFclass{Chung} &
     Chungの圧縮性モデル \\
\index{Wallis@\OFclass{Wallis}!ライブラリ}%
\index{ライブラリ!Wallis@\OFclass{Wallis}}%
 \OFclass{Wallis} &
     Wallisの圧縮性モデル \\
 \\
 \\
 \multicolumn{2}{l}{ガス種の熱物理特性 ---
\index{specie@\string\OFclass{specie}!ライブラリ}%
\index{ライブラリ!specie@\string\OFclass{specie}}%
 \OFclass{specie}} \\
 \hline
\index{adiabaticPerfectFluid@\OFclass{adiabaticPerfectFluid}!モデル}%
\index{モデル!adiabaticPerfectFluid@\OFclass{adiabaticPerfectFluid}}%
 \OFclass{adiabaticPerfectFluid} &
     断熱完全気体の状態方程式 \\
\index{icoPolynomial@\OFclass{icoPolynomial}!モデル}%
\index{モデル!icoPolynomial@\OFclass{icoPolynomial}}%
 \OFclass{icoPolynomial} &
     液体などの非圧縮性流体に対する多項式の状態方程式 \\
\index{perfectFluid@\OFclass{perfectFluid}!モデル}%
\index{モデル!perfectFluid@\OFclass{perfectFluid}}%
 \OFclass{perfectFluid} &
     完全気体の状態方程式 \\
\index{incompressiblePerfectGas@\OFclass{incompressiblePerfectGas}!モデル}%
\index{モデル!incompressiblePerfectGas@\OFclass{incompressiblePerfectGas}}%
 \OFclass{incompressiblePerfectGas} &
     一定の参照圧力を用いた非圧縮性気体の状態方程式.
     密度は温度と組成によってのみ変化する. \\
\index{rhoConst@\OFclass{rhoConst}!モデル}%
\index{モデル!rhoConst@\OFclass{rhoConst}}%
 \OFclass{rhoConst} &
     密度を一定とした状態方程式 \\
\index{eConstThermo@\OFclass{eConstThermo}!モデル}%
\index{モデル!eConstThermo@\OFclass{eConstThermo}}%
 \OFclass{eConstThermo} &
     内部エネルギ$e$とエントロピ$s$に関する一定比熱$c_{\mathrm{p}}$モデル \\
\index{hConstThermo@\OFclass{hConstThermo}!モデル}%
\index{モデル!hConstThermo@\OFclass{hConstThermo}}%
 \OFclass{hConstThermo} &
     エンタルピ$h$とエントロピ$s$に関する一定比熱$c_{\mathrm{p}}$モデル \\
\index{hPolynomialThermo@\OFclass{hPolynomialThermo}!モデル}%
\index{モデル!hPolynomialThermo@\OFclass{hPolynomialThermo}}%
 \OFclass{hPolynomialThermo} &
     $h$と$s$を評価する多項式の係数の関数により$c_{\mathrm{p}}$が評価される \\
\index{janafThermo@\OFclass{janafThermo}!モデル}%
\index{モデル!janafThermo@\OFclass{janafThermo}}%
 \OFclass{janafThermo} &
     $h$や$s$のようなJANAF熱力学テーブルの係数をもつ関数によって評価した$c_{\mathrm{p}}$ \\
\index{specieThermo@\OFclass{specieThermo}!モデル}%
\index{モデル!specieThermo@\OFclass{specieThermo}}%
 \OFclass{specieThermo} &
     $c_{\mathrm{p}}$,$h$そして/または$s$から派生するような熱物理特性 \\
\index{constTransport@\OFclass{constTransport}!モデル}%
\index{モデル!constTransport@\OFclass{constTransport}}%
 \OFclass{constTransport} &
     一定の輸送特性 \\
\index{polynomialTransport@\OFclass{polynomialTransport}!モデル}%
\index{モデル!polynomialTransport@\OFclass{polynomialTransport}}%
 \OFclass{polynomialTransport} &
     多項式に基づく温度依存輸送特性 \\
\index{sutherlandTransport@\OFclass{sutherlandTransport}!モデル}%
\index{モデル!sutherlandTransport@\OFclass{sutherlandTransport}}%
 \OFclass{sutherlandTransport} &
     温度依存輸送特性のためのSutherlandの公式 \\
 \\
 \multicolumn{2}{l}{熱物理特性の関数/表 ---
\index{thermophysicalFunctions@\string\OFclass{thermophysicalFunctions}!ライブラリ}%
\index{ライブラリ!thermophysicalFunctions@\string\OFclass{thermophysicalFunctions}}%
 \OFclass{thermophysicalFunctions}} \\
 \hline
\index{NSRDSfunctions@\OFclass{NSRDSfunctions}!モデル}%
\index{モデル!NSRDSfunctions@\OFclass{NSRDSfunctions}}%
 \OFclass{NSRDSfunctions} &
     標準参照データシステム (NSRDS) - 米国化学工学会 (AICHE) のデータ編集表 \\
\index{APIfunctions@\OFclass{APIfunctions}!モデル}%
\index{モデル!APIfunctions@\OFclass{APIfunctions}}%
 \OFclass{APIfunctions} &
     蒸気拡散のための米国石油協会 (API) の関数 \\
 \\
 \multicolumn{2}{l}{化学モデル ---
\index{chemistryModel@\string\OFclass{chemistryModel}!ライブラリ}%
\index{ライブラリ!chemistryModel@\string\OFclass{chemistryModel}}%
 \OFclass{chemistryModel}} \\
 \hline
\index{chemistryModel@\OFclass{chemistryModel}!モデル}%
\index{モデル!chemistryModel@\OFclass{chemistryModel}}%
 \OFclass{chemistryModel} &
     化学反応モデル \\
\index{chemistrySolver@\OFclass{chemistrySolver}!モデル}%
\index{モデル!chemistrySolver@\OFclass{chemistrySolver}}%
 \OFclass{chemistrySolver} &
     化学反応ソルバ \\
 \\
 \multicolumn{2}{l}{その他のライブラリ} \\
 \hline
\index{liquidProperties@\OFclass{liquidProperties}!ライブラリ}%
\index{ライブラリ!liquidProperties@\OFclass{liquidProperties}}%
 \OFclass{liquidProperties} &
     液体の熱物性 \\
\index{liquidMixtureProperties@\OFclass{liquidMixtureProperties}!ライブラリ}%
\index{ライブラリ!liquidMixtureProperties@\OFclass{liquidMixtureProperties}}%
 \OFclass{liquidMixtureProperties} &
     混合液体の熱物性 \\
\index{basicSolidThermo@\OFclass{basicSolidThermo}!ライブラリ}%
\index{ライブラリ!basicSolidThermo@\OFclass{basicSolidThermo}}%
 \OFclass{basicSolidThermo} &
     固体の熱物理モデル \\
\index{hExponentialThermo@\OFclass{hExponentialThermo}!ライブラリ}%
\index{ライブラリ!hExponentialThermo@\OFclass{hExponentialThermo}}%
 \OFclass{hExponentialThermo} &
     \OFclass{equationOfState}用にテンプレート化された指数関数物性の熱力学パッケージ \\
\index{SLGThermo@\OFclass{SLGThermo}!ライブラリ}%
\index{ライブラリ!SLGThermo@\OFclass{SLGThermo}}%
 \OFclass{SLGThermo} &
     固体・液体・気体の熱力学モデルのパッケージ \\
\index{solidChemistryModel@\OFclass{solidChemistryModel}!ライブラリ}%
\index{ライブラリ!solidChemistryModel@\OFclass{solidChemistryModel}}%
 \OFclass{solidChemistryModel} &
     熱分解を含む固体化学の熱力学モデル \\
\index{solidProperties@\OFclass{solidProperties}!ライブラリ}%
\index{ライブラリ!solidProperties@\OFclass{solidProperties}}%
 \OFclass{solidProperties} &
     固体の熱物性 \\
\index{solidMixtureProperties@\OFclass{solidMixtureProperties}!ライブラリ}%
\index{ライブラリ!solidMixtureProperties@\OFclass{solidMixtureProperties}}%
 \OFclass{solidMixtureProperties} &
     混合固体の熱物性 \\
\index{solidSpecie@\OFclass{solidSpecie}!ライブラリ}%
\index{ライブラリ!solidSpecie@\OFclass{solidSpecie}}%
 \OFclass{solidSpecie} &
     固体の反応速度と輸送のモデル \\
\index{solidThermo@\OFclass{solidThermo}!ライブラリ}%
\index{ライブラリ!solidThermo@\OFclass{solidThermo}}%
 \OFclass{solidThermo} &
     固体エネルギのモデリング
\end{longtable}
