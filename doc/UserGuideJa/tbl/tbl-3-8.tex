%#! platex UserGuideJa
\begin{longtable}{lX}
 \multicolumn{2}{l}{基本熱物理モデル ---
\index{basicThermophysicalModels@\OFemph{basicThermophysicalModels}!ライブラリ}%
\index{ライブラリ!basicThermophysicalModels@\OFemph{basicThermophysicalModels}}%
 \OFemph{basicThermophysicalModels}} \\
 \hline
\index{hPsiThermo@\OFemph{hPsiThermo}!モデル}%
\index{モデル!hPsiThermo@\OFemph{hPsiThermo}}%
 \OFemph{hPsiThermo} &
     エンタルピ$h$と圧縮率$\psi$に基づく一般熱物理モデル計算 \\
\index{ePsiThermo@\OFemph{ePsiThermo}!モデル}%
\index{モデル!ePsiThermo@\OFemph{ePsiThermo}}%
 \OFemph{ePsiThermo} &
     内部エネルギ$e$と圧縮率$\psi$に基づく一般熱物理モデル計算 \\
\index{hRhoThermo@\OFemph{hRhoThermo}!モデル}%
\index{モデル!hRhoThermo@\OFemph{hRhoThermo}}%
 \OFemph{hRhoThermo} &
     エンタルピ$h$に基づく一般熱物理モデル計算 \\
\index{pureMixture@\OFemph{pureMixture}!モデル}%
\index{モデル!pureMixture@\OFemph{pureMixture}}%
 \OFemph{pureMixture} &
     パッシブガス混合物の一般熱物理モデル計算 \\
 \\
 \multicolumn{2}{l}{化学反応モデル ---
\index{reactionThermophysicalModels@\OFemph{reactionThermophysicalModels}!ライブラリ}%
\index{ライブラリ!reactionThermophysicalModels@\OFemph{reactionThermophysicalModels}}%
 \OFemph{reactionThermophysicalModels}} \\
 \hline
\index{hPsiMixtureThermo@\OFemph{hPsiMixtureThermo}!モデル}%
\index{モデル!hPsiMixtureThermo@\OFemph{hPsiMixtureThermo}}%
 \OFemph{hPsiMixtureThermo} &
     $\psi$に基づく混合気燃焼のエンタルピ計算 \\
\index{hRhoMixtureThermo@\OFemph{hRhoMixtureThermo}!モデル}%
\index{モデル!hRhoMixtureThermo@\OFemph{hRhoMixtureThermo}}%
 \OFemph{hRhoMixtureThermo} &
     $\rho$に基づく混合気燃焼のエンタルピ計算 \\
\index{hhuMixtureThermo@\OFemph{hhuMixtureThermo}!モデル}%
\index{モデル!hhuMixtureThermo@\OFemph{hhuMixtureThermo}}%
 \OFemph{hhuMixtureThermo} &
     不燃気体と混合気のエンタルピ計算 \\
\index{homogeneousMixture@\OFemph{homogeneousMixture}!モデル}%
\index{モデル!homogeneousMixture@\OFemph{homogeneousMixture}}%
 \OFemph{homogeneousMixture} &
     標準燃料質量分率$b$に基づく混合気燃焼 \\
\index{inhomogeneousMixture@\OFemph{inhomogeneousMixture}!モデル}%
\index{モデル!inhomogeneousMixture@\OFemph{inhomogeneousMixture}}%
 \OFemph{inhomogeneousMixture} &
     $b$と総燃料質量分率$f_{\mathrm{t}}$に基づく混合気燃焼 \\
\index{veryInhomogeneousMixture@\OFemph{veryInhomogeneousMixture}!モデル}%
\index{モデル!veryInhomogeneousMixture@\OFemph{veryInhomogeneousMixture}}%
 \OFemph{veryInhomogeneousMixture} &
     $b$,$f_{\mathrm{t}}$と
     不燃燃料質量分率$f_{\mathrm{u}}$に基づく混合気燃焼 \\
\index{dieselMixture@\OFemph{dieselMixture}!モデル}%
\index{モデル!dieselMixture@\OFemph{dieselMixture}}%
 \OFemph{dieselMixture} &
     $f_{\mathrm{t}}$と$f_{\mathrm{u}}$に基づく混合気燃焼 \\
\index{basicMultiComponentMixture@\OFemph{basicMultiComponentMixture}!モデル}%
\index{モデル!basicMultiComponentMixture@\OFemph{basicMultiComponentMixture}}%
 \OFemph{basicMultiComponentMixture} &
     複数の要素に基づく基本的な混合気 \\
\index{multiComponentMixture@\OFemph{multiComponentMixture}!モデル}%
\index{モデル!multiComponentMixture@\OFemph{multiComponentMixture}}%
 \OFemph{multiComponentMixture} &
     複数の要素に基づく派生混合気 \\
\index{reactingMixture@\OFemph{reactingMixture}!モデル}%
\index{モデル!reactingMixture@\OFemph{reactingMixture}}%
 \OFemph{reactingMixture} &
     熱力学と反応スキームによる燃焼混合気 \\
\index{egrMixture@\OFemph{egrMixture}!モデル}%
\index{モデル!egrMixture@\OFemph{egrMixture}}%
 \OFemph{egrMixture} &
     排気再循環の混合気 \\
 \\
 \multicolumn{2}{l}{輻射モデル ---
\index{radiation@\OFemph{radiation}!ライブラリ}%
\index{ライブラリ!radiation@\OFemph{radiation}}%
 \OFemph{radiationModels}} \\
 \hline
\index{P1@\OFemph{P1}!ライブラリ}%
\index{ライブラリ!P1@\OFemph{P1}}%
 \OFemph{P1} &
     P1モデル \\
\index{fvDOM@\OFemph{fvDOM}!ライブラリ}%
\index{ライブラリ!fvDOM@\OFemph{fvDOM}}%
 \OFemph{fvDOM} &
     有限体積離散座標法 \\
 \\
 \multicolumn{2}{l}{層流火炎速度モデル ---
\index{laminarFlameSpeedModels@\OFemph{laminarFlameSpeedModels}!ライブラリ}%
\index{ライブラリ!laminarFlameSpeedModels@\OFemph{laminarFlameSpeedModels}}%
 \OFemph{laminarFlameSpeedModels}} \\
 \hline
\index{constLaminarFlameSpeed@\OFemph{constLaminarFlameSpeed}!モデル}%
\index{モデル!constLaminarFlameSpeed@\OFemph{constLaminarFlameSpeed}}%
 \OFemph{constLaminarFlameSpeed} &
     一定層流火炎速度 \\
\index{GuldersLaminarFlameSpeed@\OFemph{GuldersLaminarFlameSpeed}!モデル}%
\index{モデル!GuldersLaminarFlameSpeed@\OFemph{GuldersLaminarFlameSpeed}}%
 \OFemph{GuldersLaminarFlameSpeed} &
     G\"ulderの層流火炎速度モデル \\
\index{GuldersEGRLaminarFlameSpeed@\OFemph{GuldersEGRLaminarFlameSpeed}!モデル}%
\index{モデル!GuldersEGRLaminarFlameSpeed@\OFemph{GuldersEGRLaminarFlameSpeed}}%
 \OFemph{GuldersEGRLaminarFlameSpeed} &
     排気再循環モデルを伴うG\"ulderの層流火炎速度モデル \\
 \\
 \multicolumn{2}{l}{バロトロピック圧縮性モデル ---
\index{barotropicCompressibilityModels@\OFemph{barotropicCompressibilityModels}!ライブラリ}%
\index{ライブラリ!barotropicCompressibilityModels@\OFemph{barotropicCompressibilityModels}}%
 \OFemph{barotropicCompressibilityModels}} \\
 \hline
\index{linear@\OFemph{linear}!ライブラリ}%
\index{ライブラリ!linear@\OFemph{linear}}%
 \OFemph{linear} &
     線形圧縮性モデル \\
\index{Chung@\OFemph{Chung}!ライブラリ}%
\index{ライブラリ!Chung@\OFemph{Chung}}%
 \OFemph{Chung} &
     Chungの圧縮性モデル \\
\index{Wallis@\OFemph{Wallis}!ライブラリ}%
\index{ライブラリ!Wallis@\OFemph{Wallis}}%
 \OFemph{Wallis} &
     Wallisの圧縮性モデル \\
 \\
 % \multicolumn{2}{l}{液体の熱物理特性 ---
 % \index{liquids@\OFemph{liquids}!ライブラリ}%
 % \index{ライブラリ!liquids@\OFemph{liquids}}%
 % \OFemph{liquids}} \\
 % \hline
 % \index{nHeptane@\OFemph{nHeptane}!モデル}%
 % \index{モデル!nHeptane@\OFemph{nHeptane}}%
 % \OFemph{nHeptane} &
 %     ノルマルヘプタンの熱物理特性 \\
 % \index{nOctane@\OFemph{nOctane}!モデル}%
 % \index{モデル!nOctane@\OFemph{nOctane}}%
 % \OFemph{nOctane} &
 %     ノルマルオクタンの熱物理特性 \\
 % \index{nDecane@\OFemph{nDecane}!モデル}%
 % \index{モデル!nDecane@\OFemph{nDecane}}%
 % \OFemph{nDecane} &
 %     ノルマルデカンの熱物理特性 \\
 % \index{nDodecane@\OFemph{nDodecane}!モデル}%
 % \index{モデル!nDodecane@\OFemph{nDodecane}}%
 % \OFemph{nDodecane} &
 %     ノルマルドデカンの熱物理特性 \\
 % \index{isoOctane@\OFemph{isoOctane}!モデル}%
 % \index{モデル!isoOctane@\OFemph{isoOctane}}%
 % \OFemph{isoOctane} &
 %     イソオクタンの熱物理特性 \\
 % \index{diMethylEther@\OFemph{diMethylEther}!モデル}%
 % \index{モデル!diMethylEther@\OFemph{diMethylEther}}%
 % \OFemph{diMethylEther} &
 %     ジメチルエーテルの熱物理特性 \\
 % \index{diEthylEther@\OFemph{diEthylEther}!モデル}%
 % \index{モデル!diEthylEther@\OFemph{diEthylEther}}%
 % \OFemph{diEthylEther} &
 %     ジエチルエーテルの熱物理特性 \\
 % \index{water@\OFemph{water}!モデル}%
 % \index{モデル!water@\OFemph{water}}%
 % \OFemph{water} &
 %     水の熱物理特性 \\
 % \\
 \multicolumn{2}{l}{ガス種の熱物理特性 ---
\index{specie@\OFemph{specie}!ライブラリ}%
\index{ライブラリ!specie@\OFemph{specie}}%
 \OFemph{specie}} \\
 \hline
\index{icoPolynomial@\OFemph{icoPolynomial}!モデル}%
\index{モデル!icoPolynomial@\OFemph{icoPolynomial}}%
 \OFemph{icoPolynomial} &
     液体などの非圧縮性流体に対する多項式の状態方程式 \\
\index{perfectGas@\OFemph{perfectGas}!モデル}%
\index{モデル!perfectGas@\OFemph{perfectGas}}%
 \OFemph{perfectGas} &
     理想気体に対する状態方程式 \\
\index{eConstThermo@\OFemph{eConstThermo}!モデル}%
\index{モデル!eConstThermo@\OFemph{eConstThermo}}%
 \OFemph{eConstThermo} &
     内部エネルギ$e$とエントロピ$s$に関する一定比熱$c_{\mathrm{p}}$モデル \\
\index{hConstThermo@\OFemph{hConstThermo}!モデル}%
\index{モデル!hConstThermo@\OFemph{hConstThermo}}%
 \OFemph{hConstThermo} &
     エンタルピ$h$とエントロピ$s$に関する一定比熱$c_{\mathrm{p}}$モデル \\
\index{hPolynomialThermo@\OFemph{hPolynomialThermo}!モデル}%
\index{モデル!hPolynomialThermo@\OFemph{hPolynomialThermo}}%
 \OFemph{hPolynomialThermo} &
     $h$と$s$を評価する多項式の係数の関数により$c_{\mathrm{p}}$が評価される \\
\index{janafThermo@\OFemph{janafThermo}!モデル}%
\index{モデル!janafThermo@\OFemph{janafThermo}}%
 \OFemph{janafThermo} &
     $h$や$s$のようなJANAF熱力学テーブルの係数をもつ関数によって評価した$c_{\mathrm{p}}$ \\
\index{specieThermo@\OFemph{specieThermo}!モデル}%
\index{モデル!specieThermo@\OFemph{specieThermo}}%
 \OFemph{specieThermo} &
     $c_{\mathrm{p}}$,$h$そして/または$s$から派生するような熱物理特性 \\
\index{constTransport@\OFemph{constTransport}!モデル}%
\index{モデル!constTransport@\OFemph{constTransport}}%
 \OFemph{constTransport} &
     一定の輸送特性 \\
\index{polynomialTransport@\OFemph{polynomialTransport}!モデル}%
\index{モデル!polynomialTransport@\OFemph{polynomialTransport}}%
 \OFemph{polynomialTransport} &
     多項式に基づく温度依存輸送特性 \\
\index{sutherlandTransport@\OFemph{sutherlandTransport}!モデル}%
\index{モデル!sutherlandTransport@\OFemph{sutherlandTransport}}%
 \OFemph{sutherlandTransport} &
     温度依存輸送特性のためのサザーランドの公式 \\
 \\
 \multicolumn{2}{l}{熱物理特性の関数/表 ---
\index{thermophysicalFunctions@\OFemph{thermophysicalFunctions}!ライブラリ}%
\index{ライブラリ!thermophysicalFunctions@\OFemph{thermophysicalFunctions}}%
 \OFemph{thermophysicalFunctions}} \\
 \hline
\index{NSRDSfunctions@\OFemph{NSRDSfunctions}!モデル}%
\index{モデル!NSRDSfunctions@\OFemph{NSRDSfunctions}}%
 \OFemph{NSRDSfunctions} &
     標準参照データシステム (NSRDS) - 米国化学工学会 (AICHE) のデータ編集表 \\
\index{APIfunctions@\OFemph{APIfunctions}!モデル}%
\index{モデル!APIfunctions@\OFemph{APIfunctions}}%
 \OFemph{APIfunctions} &
     蒸気拡散のための米国石油協会 (API) の関数 \\
 \\
 \multicolumn{2}{l}{確率密度関数 ---
\index{pdf@\OFemph{pdf}!ライブラリ}%
\index{ライブラリ!pdf@\OFemph{pdf}}%
 \OFemph{pdf}} \\
 \hline
\index{RosinRammler@\OFemph{RosinRammler}!モデル}%
\index{モデル!RosinRammler@\OFemph{RosinRammler}}%
 \OFemph{RosinRammler} &
     ロジン・ラムラー分布 \\
\index{normal@\OFemph{normal}!モデル}%
\index{モデル!normal@\OFemph{normal}}%
 \OFemph{normal} &
     正規分布 \\
\index{uniform@\OFemph{uniform}!モデル}%
\index{モデル!uniform@\OFemph{uniform}}%
 \OFemph{uniform} &
     一様分布 \\
\index{exponential@\OFemph{exponential}!モデル}%
\index{モデル!exponential@\OFemph{exponential}}%
 \OFemph{exponential} &
     指数分布 \\
\index{general@\OFemph{general}!モデル}%
\index{モデル!general@\OFemph{general}}%
 \OFemph{general} &
     一般化分布 \\
 \\
 \multicolumn{2}{l}{化学モデル ---
\index{chemistryModel@\OFemph{chemistryModel}!ライブラリ}%
\index{ライブラリ!chemistryModel@\OFemph{chemistryModel}}%
 \OFemph{chemistryModel}} \\
\index{chemistryModel@\OFemph{chemistryModel}!モデル}%
\index{モデル!chemistryModel@\OFemph{chemistryModel}}%
 \OFemph{chemistryModel} &
     化学反応モデル \\
\index{chemistrySolver@\OFemph{chemistrySolver}!モデル}%
\index{モデル!chemistrySolver@\OFemph{chemistrySolver}}%
 \OFemph{chemistrySolver} &
     化学反応ソルバ
 \\
 \multicolumn{2}{l}{その他のライブラリ} \\
\index{liquids@\OFemph{liquids}!ライブラリ}%
\index{ライブラリ!liquids@\OFemph{liquids}}%
 \OFemph{liquids} &
     液体の熱物性 \\
\index{liquidMixture@\OFemph{liquidMixture}!ライブラリ}%
\index{ライブラリ!liquidMixture@\OFemph{liquidMixture}}%
 \OFemph{liquidMixture} &
     混合液体の熱物性 \\
\index{solids@\OFemph{solids}!ライブラリ}%
\index{ライブラリ!solids@\OFemph{solids}}%
 \OFemph{solids} &
     固体の熱物性 \\
\index{solidMixture@\OFemph{solidMixture}!ライブラリ}%
\index{ライブラリ!solidMixture@\OFemph{solidMixture}}%
 \OFemph{solidMixture} &
     混合固体の熱物性 \\
\end{longtable}
