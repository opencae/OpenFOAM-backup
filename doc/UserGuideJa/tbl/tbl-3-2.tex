%#! platex UserGuideJa
\begin{tabularx}{\textwidth}{lX}
 主なパス & \\
 \hline
\index{WM PROJECT INST DIR@\OFenv{WM\_PROJECT\_INST\_DIR}!かんきょうへんすう@環境変数}%
\index{かんきょうへんすう@環境変数!WM PROJECT INST DIR@\OFenv{WM\_PROJECT\_INST\_DIR}}%
 \OFenv{\$WM\_PROJECT\_INST\_DIR}
 & インストールディレクトリへのフルパス,例:\OFpath{\$HOME/OpenFOAM} \\
\index{WM PROJECT@\OFenv{WM\_PROJECT}!かんきょうへんすう@環境変数}%
\index{かんきょうへんすう@環境変数!WM PROJECT@\OFenv{WM\_PROJECT}}%
 \OFenv{\$WM\_PROJECT}
 & コンパイルされたプロジェクトの名前:\texttt{OpenFOAM} \\
\index{WM PROJECT VERSION@\OFenv{WM\_PROJECT\_VERSION}!かんきょうへんすう@環境変数}%
\index{かんきょうへんすう@環境変数!WM PROJECT VERSION@\OFenv{WM\_PROJECT\_VERSION}}%
 \OFenv{\$WM\_PROJECT\_VERSION}
 & コンパイルされたプロジェクトのバージョン:\texttt{1.6} \\
\index{WM PROJECT DIR@\OFenv{WM\_PROJECT\_DIR}!かんきょうへんすう@環境変数}%
\index{かんきょうへんすう@環境変数!WM PROJECT DIR@\OFenv{WM\_PROJECT\_DIR}}%
 \OFenv{\$WM\_PROJECT\_DIR}
 & OpenFOAMのバイナリ実行ファイル置き場へのフルパス,\hfil\break
     例:\OFpath{\$HOME/OpenFOAM/OpenFOAM-1.6} \\
\index{WM PROJECT USER DIR@\OFenv{WM\_PROJECT\_USER\_DIR}!かんきょうへんすう@環境変数}%
\index{かんきょうへんすう@環境変数!WM PROJECT USER DIR@\OFenv{WM\_PROJECT\_USER\_DIR}}%
 \OFenv{\$WM\_PROJECT\_USER\_DIR}
 & ユーザのバイナリ実行ファイル置き場へのフルパス,\hfil\break
     例:\OFpath{\$HOME/OpenFOAM/\${USER}-1.6} \\
 \\
 その他のパスと設定 & \\
 \hline
\index{WM ARCH@\OFenv{WM\_ARCH}!かんきょうへんすう@環境変数}%
\index{かんきょうへんすう@環境変数!WM ARCH@\OFenv{WM\_ARCH}}%
 \OFenv{\$WM\_ARCH}
 & マシン構造:
     \texttt{cray decAlpha dec ibm linux linuxPPC sgi3 sgi32\hfil\break
     sgi64 sgiN32 solaris sx4 t3d} \\
\index{WM COMPILER@\OFenv{WM\_COMPILER}!かんきょうへんすう@環境変数}%
\index{かんきょうへんすう@環境変数!WM COMPILER@\OFenv{WM\_COMPILER}}%
 \OFenv{\$WM\_COMPILER}
 & 使用するコンパイラ:\texttt{Gcc3} - \textsf{gcc} 4.3.3, \texttt{KAI} - \textsf{KAI} \\
\index{WM COMPILER DIR@\OFenv{WM\_COMPILER\_DIR}!かんきょうへんすう@環境変数}%
\index{かんきょうへんすう@環境変数!WM COMPILER DIR@\OFenv{WM\_COMPILER\_DIR}}%
 \OFenv{\$WM\_COMPILER\_DIR}
 & コンパイラインストールディレクトリ \\
\index{WM COMPILER BIN@\OFenv{WM\_COMPILER\_BIN}!かんきょうへんすう@環境変数}%
\index{かんきょうへんすう@環境変数!WM COMPILER BIN@\OFenv{WM\_COMPILER\_BIN}}%
 \OFenv{\$WM\_COMPILER\_BIN}
 & コンパイラインストールバイナリ:\OFpath{\$WM\_COMPILER\_BIN/bin} \\
\index{WM COMPILER LIB@\OFenv{WM\_COMPILER\_LIB}!かんきょうへんすう@環境変数}%
\index{かんきょうへんすう@環境変数!WM COMPILER LIB@\OFenv{WM\_COMPILER\_LIB}}%
 \OFenv{\$WM\_COMPILER\_LIB}
 & コンパイラインストールライブラリ:\OFpath{\$WM\_COMPILER\_BIN/lib} \\
\index{WM COMPILE OPTION@\OFenv{WM\_COMPILE\_OPTION}!かんきょうへんすう@環境変数}%
\index{かんきょうへんすう@環境変数!WM COMPILE OPTION@\OFenv{WM\_COMPILE\_OPTION}}%
 \OFenv{\$WM\_COMPILE\_OPTION}
 & コンパイルオプション:\texttt{Debug} - debugging, \texttt{Opt} optimisation. \\
\index{WM DIR@\OFenv{WM\_DIR}!かんきょうへんすう@環境変数}%
\index{かんきょうへんすう@環境変数!WM DIR@\OFenv{WM\_DIR}}%
 \OFenv{\$WM\_DIR}
 & wmakeディレクトリのフルパス \\
\index{WM JAVAC OPTION@\OFenv{WM\_JAVAC\_OPTION}!かんきょうへんすう@環境変数}%
\index{かんきょうへんすう@環境変数!WM JAVAC OPTION@\OFenv{WM\_JAVAC\_OPTION}}%
 \OFenv{\$WM\_JAVAC\_OPTION}
 & JAVAのためのコンパイルオプション:\texttt{Debug} - debugging, \texttt{Opt} optimisation. \\
\index{WM LINK LANGUAGE@\OFenv{WM\_LINK\_LANGUAGE}!かんきょうへんすう@環境変数}%
\index{かんきょうへんすう@環境変数!WM LINK LANGUAGE@\OFenv{WM\_LINK\_LANGUAGE}}%
 \OFenv{\$WM\_LINK\_LANGUAGE}
 & ライブラリや実行ファイルのリンクに使うコンパイラ.
     多言語プロジェクトにおいて\OFenv{\$WM\_LINK\_LANGUAGE}は主要言語を決める. \\
\index{WM MPLIB@\OFenv{WM\_MPLIB}!かんきょうへんすう@環境変数}%
\index{かんきょうへんすう@環境変数!WM MPLIB@\OFenv{WM\_MPLIB}}%
 \OFenv{\$WM\_MPLIB}
 & 並列通信ライブラリ:\texttt{LAM},\texttt{MPI},\texttt{MPICH},\texttt{PVM} \\
\index{WM OPTIONS@\OFenv{WM\_OPTIONS}!かんきょうへんすう@環境変数}%
\index{かんきょうへんすう@環境変数!WM OPTIONS@\OFenv{WM\_OPTIONS}}%
 \OFenv{\$WM\_OPTIONS}
 & $=$ \OFenv{\$WM\_ARCH\$WM\_COMPILER...}\hfill\break
     \null\hfill\OFenv{...\$WM\_COMPILE\_OPTION\$WM\_MPLIB},\break
     例:\texttt{linuxGcc3OptMPICH} \\
\index{WM PROJECT LANGUAGE@\OFenv{WM\_PROJECT\_LANGUAGE}!かんきょうへんすう@環境変数}%
\index{かんきょうへんすう@環境変数!WM PROJECT LANGUAGE@\OFenv{WM\_PROJECT\_LANGUAGE}}%
 \OFenv{\$WM\_PROJECT\_LANGUAGE}
 & プロジェクトのプログラミング言語,例:\texttt{c++} \\
\index{WM SHELL@\OFenv{WM\_SHELL}!かんきょうへんすう@環境変数}%
\index{かんきょうへんすう@環境変数!WM SHELL@\OFenv{WM\_SHELL}}%
 \OFenv{\$WM\_SHELL}
 & wmakeスクリプトに使うシェル:\texttt{bash},\texttt{csh},\texttt{ksh},\texttt{tcsh} \\
 \hline
\end{tabularx}
