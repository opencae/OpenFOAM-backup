%#! platex UserGuideJa
\begin{tabularx}{\textwidth}{lXp{8zw}}
 種類 & 物理量$\phi$に対して与える条件 & 与えるデータ \\
 \hline
\index{fixedValue@\string\OFboundary{fixedValue}!きょうかいじょうけん@境界条件}%
\index{きょうかいじょうけん@境界条件!fixedValue@\string\OFboundary{fixedValue}}%
 \OFboundary{fixedValue} & $\phi$の値を指定 & \OFkeyword{value} \\
\index{fixedGradient@\string\OFboundary{fixedGradient}!きょうかいじょうけん@境界条件}%
\index{きょうかいじょうけん@境界条件!fixedGradient@\string\OFboundary{fixedGradient}}%
 \OFboundary{fixedGradient} & $\phi$の勾配を指定 & \OFkeyword{gradient} \\
\index{zeroGradient@\string\OFboundary{zeroGradient}!きょうかいじょうけん@境界条件}%
\index{きょうかいじょうけん@境界条件!zeroGradient@\string\OFboundary{zeroGradient}}%
 \OFboundary{zeroGradient} & $\phi$の勾配が$0$ & -- \\
\index{calculated@\string\OFboundary{calculated}!きょうかいじょうけん@境界条件}%
\index{きょうかいじょうけん@境界条件!calculated@\string\OFboundary{calculated}}%
 \OFboundary{calculated} & $\phi$の境界条件が他の物理量から決まる & -- \\
\index{mixed@\string\OFboundary{mixed}!きょうかいじょうけん@境界条件}%
\index{きょうかいじょうけん@境界条件!mixed@\string\OFboundary{mixed}}%
 \OFboundary{mixed} & \OFboundary{fixedValue}と\OFboundary{fixedGradient}の組み合わせ. 
     \OFkeyword{valueFraction}に依存する条件 &
         \OFkeyword{refValue},\hfil\break
\index{refGradient@\string\OFkeyword{refGradient}!キーワード}%
\index{キーワード!refGradient@\string\OFkeyword{refGradient}}%
         \OFkeyword{refGradient},\hfil\break
\index{valueFraction@\string\OFkeyword{valueFraction}!キーワード}%
\index{キーワード!valueFraction@\string\OFkeyword{valueFraction}}%
         \OFkeyword{valueFraction},\hfil\break
\index{value@\string\OFkeyword{value}!キーワード}%
\index{キーワード!value@\string\OFkeyword{value}}%
         \OFkeyword{value} \\
\index{directionMixed@\string\OFboundary{directionMixed}!きょうかいじょうけん@境界条件}%
\index{きょうかいじょうけん@境界条件!directionMixed@\string\OFboundary{directionMixed}}%
 \OFboundary{directionMixed} &
     \OFrevision*{要再訳}%
     テンソル型の\OFkeyword{valueFraction}を用いる\OFkeyword{mixed}条件.
     例えば法線方向と接線方向で異なる取り扱いを行う場合
      &
         \OFkeyword{refValue},\hfil\break
         \OFkeyword{refGradient},\hfil\break
         \OFkeyword{valueFraction},\hfil\break
         \OFkeyword{value} \\
 \hline
\end{tabularx}
