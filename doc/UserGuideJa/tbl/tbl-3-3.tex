%#! platex UserGuideJa
\begin{tabularx}{\textwidth}{lX}
 \multicolumn{2}{l}{ハイレベルデバッグスイッチ - サブディクショナリ \OFdictionary{DebugSwitches}} \\
 \hline
 \texttt{level} & OpenFOAMのデバッグメッセージの全体のレベル - - 0, 1, 2の3レベル \\
 \texttt{lduMatrix} & 実行中のソルバの収束メッセージ - 0, 1, 2の3レベル \\
 \\
 \multicolumn{2}{l}{最適化スイッチ - サブディクショナリ \OFdictionary{OptimisationSwitches}} \\
 \hline
 \index{fileModificationSkew@\texttt{fileModificationSkew}!キーワード}%
 \index{キーワード!fileModificationSkew@\texttt{fileModificationSkew}}%
 \texttt{fileModificationSkew} & NFSの上でNFSのアップデートの最大遅れと
     OpenFOAM実行のための差分クロックより長く設定すべき時間(秒). \\
 \index{fileModificationChecking@\texttt{fileModificationChecking}!キーワード}%
 \index{キーワード!fileModificationChecking@\texttt{fileModificationChecking}}%
 \texttt{fileModificationChecking} & シミュレーション中にファイルが変更されたか
     どうかをチェックする方法であり,
	 \index{timeStamp@\OFkeyword{timeStamp}!キーワードエントリ}%
	 \index{キーワードエントリ!timeStamp@\OFkeyword{timeStamp}}%
	 \OFkeyword{timeStamp}を読む,
     または
	 \index{inotify@\OFkeyword{inotify}!キーワードエントリ}%
	 \index{キーワードエントリ!inotify@\OFkeyword{inotify}}%
	 \OFkeyword{inotify}(通知しない),
     あるいはマスタ・ノードに存在するデータのみを読み込む
	 \index{timeStampMaster@\OFkeyword{timeStampMaster}!キーワードエントリ}%
	 \index{キーワードエントリ!timeStampMaster@\OFkeyword{timeStampMaster}}%
	 \OFkeyword{timeStampMaster},
     \index{inotifyMaster@\OFkeyword{inotifyMaster}!キーワードエントリ}%
     \index{キーワードエントリ!inotifyMaster@\OFkeyword{inotifyMaster}}%
	 \OFkeyword{inotifyMaster}があります. \\
\index{commsType@\texttt{commsType}!キーワード}%
\index{キーワード!commsType@\texttt{commsType}}%
\texttt{commsType} & 並列計算の通信方法.\OFkeyword{nonBlocking},
     \OFkeyword{scheduled},\OFkeyword{blocking}. \\
	 \index{nonBlocking@\OFkeyword{nonBlocking}!キーワードエントリ}%
	 \index{キーワードエントリ!nonBlocking@\OFkeyword{nonBlocking}}%
	 \index{scheduled@\OFkeyword{scheduled}!キーワードエントリ}%
	 \index{キーワードエントリ!scheduled@\OFkeyword{scheduled}}%
	 \index{blocking@\OFkeyword{blocking}!キーワードエントリ}%
	 \index{キーワードエントリ!blocking@\OFkeyword{blocking}}%
\index{floatTransfer@\texttt{floatTransfer}!キーワード}%
\index{キーワード!floatTransfer@\texttt{floatTransfer}}%
\texttt{floatTransfer} & \OFkeyword{1}とすると,
     転送の前に数値を\OFkeyword{float}の精度に丸めます.デフォルトは\OFkeyword{0}です. \\
 \texttt{nProcsSimpleSum} & 並列処理のために全領域の和を最適化します.
     階層和は線形和(デフォルトで16)よりよく機能し,
     プロセッサの数を設定します. \\
 \hline
\end{tabularx}
