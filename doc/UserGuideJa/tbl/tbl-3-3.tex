%#! platex UserGuideJa
\begin{tabularx}{\textwidth}{lX}
 \multicolumn{2}{l}{ハイレベルデバッグスイッチ - サブディクショナリ \OFdictionary{DebugSwitches}} \\
 \hline
 \OFkeyword{level} & OpenFOAMのデバッグメッセージの全体のレベル - - 0, 1, 2の3レベル \\
 \OFkeyword{lduMatrix} & 実行中のソルバの収束メッセージ - 0, 1, 2の3レベル \\
 \\
 \multicolumn{2}{l}{最適化スイッチ - サブディクショナリ \OFdictionary{OptimisationSwitches}} \\
 \hline
\index{fileModificationSkew@\OFkeyword{fileModificationSkew}!キーワード}%
\index{キーワード!fileModificationSkew@\OFkeyword{fileModificationSkew}}%
 \OFkeyword{fileModificationSkew} & NFSの上でNFSのアップデートの最大遅れと
     OpenFOAM実行のための差分クロックより長く設定すべき時間(秒). \\
\index{fileModificationChecking@\OFkeyword{fileModificationChecking}!キーワード}%
\index{キーワード!fileModificationChecking@\OFkeyword{fileModificationChecking}}%
 \OFkeyword{fileModificationChecking} & シミュレーション中にファイルが変更されたか
     どうかをチェックする方法であり,
\index{timeStamp@\OFkeyword{timeStamp}!キーワードエントリ}%
\index{キーワードエントリ!timeStamp@\OFkeyword{timeStamp}}%
     \OFkeyword{timeStamp}を読む,
     または
\index{inotify@\OFkeyword{inotify}!キーワードエントリ}%
\index{キーワードエントリ!inotify@\OFkeyword{inotify}}%
     \OFkeyword{inotify}(通知しない),
     あるいはマスタ・ノードに存在するデータのみを読み込む
\index{timeStampMaster@\OFkeyword{timeStampMaster}!キーワードエントリ}%
\index{キーワードエントリ!timeStampMaster@\OFkeyword{timeStampMaster}}%
     \OFkeyword{timeStampMaster},
\index{inotifyMaster@\OFkeyword{inotifyMaster}!キーワードエントリ}%
\index{キーワードエントリ!inotifyMaster@\OFkeyword{inotifyMaster}}%
     \OFkeyword{inotifyMaster}があります. \\
\index{commsType@\OFkeyword{commsType}!キーワード}%
\index{キーワード!commsType@\OFkeyword{commsType}}%
 \OFkeyword{commsType} & 並列計算の通信方法.
\index{nonBlocking@\OFkeyword{nonBlocking}!キーワードエントリ}%
\index{キーワードエントリ!nonBlocking@\OFkeyword{nonBlocking}}%
     \OFkeyword{nonBlocking},
\index{scheduled@\OFkeyword{scheduled}!キーワードエントリ}%
\index{キーワードエントリ!scheduled@\OFkeyword{scheduled}}%
     \OFkeyword{scheduled},
\index{blocking@\OFkeyword{blocking}!キーワードエントリ}%
\index{キーワードエントリ!blocking@\OFkeyword{blocking}}%
     \OFkeyword{blocking}. \\
\index{floatTransfer@\OFkeyword{floatTransfer}!キーワード}%
\index{キーワード!floatTransfer@\OFkeyword{floatTransfer}}%
 \OFkeyword{floatTransfer} & \OFkeyword{1}とすると,
     転送の前に数値を\OFkeyword{float}の精度に丸めます.デフォルトは\OFkeyword{0}です. \\
 \OFkeyword{nProcsSimpleSum} & 並列処理のために全領域の和を最適化します.
     階層和は線形和(デフォルトで16)よりよく機能し,
     プロセッサの数を設定します. \\
 \hline
\end{tabularx}
