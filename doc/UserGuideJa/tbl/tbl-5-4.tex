%#! platex UserGuideJa
\begin{tabularx}{\textheight}{lXp{9zw}}
 \OFboundary{fixedValue}から派生 & 意味 & 指定するデータ \\
 \hline
\index{movingWallVelocity@\OFboundary{movingWallVelocity}!きょうかいじょうけん@境界条件}%
\index{きょうかいじょうけん@境界条件!movingWallVelocity@\OFboundary{movingWallVelocity}}%
 \OFboundary{movingWallVelocity} &
     ノーマルパッチの値を置き換えるのでパッチのフラックスは$0$ & \OFkeyword{value} \\
\index{pressureInletVelocity@\OFboundary{pressureInletVelocity}!きょうかいじょうけん@境界条件}%
\index{きょうかいじょうけん@境界条件!pressureInletVelocity@\OFboundary{pressureInletVelocity}}%
 \OFboundary{pressureInletVelocity} &
     流入口の$p$が分かっているとき,$\bm{U}$は,フラックスから評価され,
     パッチはノーマル. & \OFkeyword{value} \\
\index{pressureDirectedInletVelocity@\OFboundary{pressureDirectedInletVelocity}!きょうかいじょうけん@境界条件}%
\index{きょうかいじょうけん@境界条件!pressureDirectedInletVelocity@\OFboundary{pressureDirectedInletVelocity}}%
 \OFboundary{pressureDirectedInletVelocity} &
     流入口の$p$が分かっているとき,$\bm{U}$は,
     \OFkeyword{inletDirection}のフラックスから計算される. &
         \OFkeyword{value},\OFkeyword{inletDirection} \\
\index{surfaceNormalFixedValue@\OFboundary{surfaceNormalFixedValue}!きょうかいじょうけん@境界条件}%
\index{きょうかいじょうけん@境界条件!surfaceNormalFixedValue@\OFboundary{surfaceNormalFixedValue}}%
 \OFboundary{surfaceNormalFixedValue} &
     大きさによって,ベクトル境界条件をノーマルパッチに指定します.
     ベクトルの+veはドメインを指す. & \OFkeyword{value} \\
\index{totalPressure@\OFboundary{totalPressure}!きょうかいじょうけん@境界条件}%
\index{きょうかいじょうけん@境界条件!totalPressure@\OFboundary{totalPressure}}%
 \OFboundary{totalPressure} &
     全圧$p_{0} = p + \frac{1}{2}\rho|\bm{U}|^{2}$は固定.
     $\bm{U}$が変わるとそれに従い$p$も調整される. & \OFkeyword{p0} \\
\index{turbulentInlet@\OFboundary{turbulentInlet}!きょうかいじょうけん@境界条件}%
\index{きょうかいじょうけん@境界条件!turbulentInlet@\OFboundary{turbulentInlet}}%
 \OFboundary{turbulentInlet} &
     平均値のスケールに基づく変動変数について計算する &
         \OFkeyword{referenceField}, \OFkeyword{fluctuationScale} \\
 \\
 \OFboundary{fixedGradient}/\OFboundary{zeroGradient}から派生 \\
 \hline
\index{fluxCorrectedVelocity@\OFboundary{fluxCorrectedVelocity}!きょうかいじょうけん@境界条件}%
\index{きょうかいじょうけん@境界条件!fluxCorrectedVelocity@\OFboundary{fluxCorrectedVelocity}}%
 \OFboundary{fluxCorrectedVelocity} &
     フラックスから流入口の$\bm{U}$の法線成分を計算する &
         \OFkeyword{value} \\
\index{wallBuoyantPressure@\OFboundary{wallBuoyantPressure}!きょうかいじょうけん@境界条件}%
\index{きょうかいじょうけん@境界条件!wallBuoyantPressure@\OFboundary{wallBuoyantPressure}}%
 \OFboundary{wallBuoyantPressure} &
     気圧勾配に基づく\OFboundary{fixedGradient}圧を設定する & --- \\
 \\
 \OFboundary{mixed}から派生 \\
 \hline
\index{inletOutlet@\OFboundary{inletOutlet}!きょうかいじょうけん@境界条件}%
\index{きょうかいじょうけん@境界条件!inletOutlet@\OFboundary{inletOutlet}}%
 \OFboundary{inletOutlet} &
     $\bm{U}$の向きによって\OFboundary{fixedValue}と
     \OFboundary{zeroGradient}の間で$\bm{U}$と$p$を切り替える &
         \OFkeyword{inletValue},\OFkeyword{value} \\
\index{outletInlet@\OFboundary{outletInlet}!きょうかいじょうけん@境界条件}%
\index{きょうかいじょうけん@境界条件!outletInlet@\OFboundary{outletInlet}}%
 \OFboundary{outletInlet} &
     $\bm{U}$の向きによって\OFboundary{fixedValue}と
     \OFboundary{zeroGradient}の間で$\bm{U}$と$p$を切り替える &
         \OFkeyword{outletValue},\OFkeyword{value} \\
 \OFboundary{pressureInletOutletVelocity} &
     \OFboundary{pressureInletVelocity}と
     \OFboundary{inletOutlet}の組み合わせ & \OFkeyword{value} \\
 \OFboundary{pressureDirectedInletOutletVelocity} &
     \OFboundary{pressureDirectedInletVelocity}と\OFboundary{inletOutlet}の組み合わせ &
         \OFkeyword{value},\OFkeyword{inletDirection} \\
\index{pressureTransmissive@\OFboundary{pressureTransmissive}!きょうかいじょうけん@境界条件}%
\index{きょうかいじょうけん@境界条件!pressureTransmissive@\OFboundary{pressureTransmissive}}%
 \OFboundary{pressureTransmissive} &
     周囲の圧力$p_{\infty}$に超音速圧縮波を伝える & \OFkeyword{pInf} \\
\index{supersonicFreeStream@\OFboundary{supersonicFreeStream}!きょうかいじょうけん@境界条件}%
\index{きょうかいじょうけん@境界条件!supersonicFreeStream@\OFboundary{supersonicFreeStream}}%
 \OFboundary{supersonicFreeStream} &
     斜めの衝撃を$p_{\infty}$,$T_{\infty}$,$U_{\infty}$の環境に伝える &
         \OFkeyword{pInf},\OFkeyword{TInf},\OFkeyword{UInf} \\
 その他 \\
 \hline
\index{slip@\OFboundary{slip}!きょうかいじょうけん@境界条件}%
\index{きょうかいじょうけん@境界条件!slip@\OFboundary{slip}}%
 \OFboundary{slip} & $\phi$がスカラなら\OFboundary{zeroGradient},
     $\phi$がベクトルなら法線成分は\OFboundary{fixedValue 0}で,
     接線成分は\OFboundary{zeroGradient} & --- \\
\index{partialSlip@\OFboundary{partialSlip}!きょうかいじょうけん@境界条件}%
\index{きょうかいじょうけん@境界条件!partialSlip@\OFboundary{partialSlip}}%
 \OFboundary{partialSlip} &
     混合\OFboundary{zeroGradient}/\OFboundary{slip}条件は
     \OFkeyword{valueFraction}による.\OFboundary{slip}ならば$1$. &
         \OFkeyword{valueFraction} \\
 \hline
\end{tabularx}
