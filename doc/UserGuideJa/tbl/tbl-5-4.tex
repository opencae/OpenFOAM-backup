%#! platex UserGuideJa
\begin{tabularx}{\textheight}{lXp{9zw}}
 \OFboundary{fixedValue}から派生 & 意味 & 指定するデータ \\
 \hline
 \tblstrut
\index{movingWallVelocity@\string\OFboundary{movingWallVelocity}!きょうかいじょうけん@境界条件}%
\index{きょうかいじょうけん@境界条件!movingWallVelocity@\string\OFboundary{movingWallVelocity}}%
 \OFboundary{movingWallVelocity} &
     パッチでのフラックスが$0$となるよう,パッチでの値の法線方向を置き換える. & \OFkeyword{value} \\
\index{pressureInletVelocity@\string\OFboundary{pressureInletVelocity}!きょうかいじょうけん@境界条件}%
\index{きょうかいじょうけん@境界条件!pressureInletVelocity@\string\OFboundary{pressureInletVelocity}}%
 \OFboundary{pressureInletVelocity} &
     流入口の$p$が分かっているとき,$\bm{U}$の値はフラックスから,
     法線方向はパッチから評価される. & \OFkeyword{value} \\
\index{pressureDirectedInletVelocity@\string\OFboundary{pressureDirectedInletVelocity}!きょうかいじょうけん@境界条件}%
\index{きょうかいじょうけん@境界条件!pressureDirectedInletVelocity@\string\OFboundary{pressureDirectedInletVelocity}}%
 \OFboundary{pressureDirectedInletVelocity} &
     流入口の$p$が分かっているとき,$\bm{U}$の値はフラックスから計算され,
     \OFkeyword{inletDirection}の方向となる. &
         \OFkeyword{value},\OFkeyword{inletDirection} \\
\index{surfaceNormalFixedValue@\string\OFboundary{surfaceNormalFixedValue}!きょうかいじょうけん@境界条件}%
\index{きょうかいじょうけん@境界条件!surfaceNormalFixedValue@\string\OFboundary{surfaceNormalFixedValue}}%
 \OFboundary{surfaceNormalFixedValue} &
     パッチの法線方向にベクトル型境界条件をその大きさによって指定する.
     領域の外側を示す方向が正. & \OFkeyword{value} \\
\index{totalPressure@\string\OFboundary{totalPressure}!きょうかいじょうけん@境界条件}%
\index{きょうかいじょうけん@境界条件!totalPressure@\string\OFboundary{totalPressure}}%
 \OFboundary{totalPressure} &
     全圧$p_{0} = p + \frac{1}{2}\rho|\bm{U}|^{2}$は固定.
     $\bm{U}$が変わるとそれに従い$p$も調整される. & \OFkeyword{p0} \\
\index{turbulentInlet@\string\OFboundary{turbulentInlet}!きょうかいじょうけん@境界条件}%
\index{きょうかいじょうけん@境界条件!turbulentInlet@\string\OFboundary{turbulentInlet}}%
 \OFboundary{turbulentInlet} &
     平均値のスケールに基づく変動変数について計算する &
         \OFkeyword{referenceField}, \OFkeyword{fluctuationScale} \\
 \\
 \OFboundary{fixedGradient}/\OFboundary{zeroGradient}から派生 \\
 \hline
 \tblstrut
\index{fluxCorrectedVelocity@\string\OFboundary{fluxCorrectedVelocity}!きょうかいじょうけん@境界条件}%
\index{きょうかいじょうけん@境界条件!fluxCorrectedVelocity@\string\OFboundary{fluxCorrectedVelocity}}%
 \OFboundary{fluxCorrectedVelocity} &
     フラックスから流入口の$\bm{U}$の法線成分を計算する &
         \OFkeyword{value} \\
\index{buoyantPressure@\string\OFboundary{buoyantPressure}!きょうかいじょうけん@境界条件}%
\index{きょうかいじょうけん@境界条件!buoyantPressure@\string\OFboundary{buoyantPressure}}%
 \OFboundary{buoyantPressure} &
     気圧勾配に基づく圧力に\OFboundary{fixedGradient}設定する & --- \\
 \\
 \OFboundary{mixed}から派生 \\
 \hline
 \tblstrut
\index{inletOutlet@\string\OFboundary{inletOutlet}!きょうかいじょうけん@境界条件}%
\index{きょうかいじょうけん@境界条件!inletOutlet@\string\OFboundary{inletOutlet}}%
 \OFboundary{inletOutlet} &
     $\bm{U}$の向きによって\OFboundary{fixedValue}と
     \OFboundary{zeroGradient}の間で$\bm{U}$と$p$を切り替える &
         \OFkeyword{inletValue},\OFkeyword{value} \\
\index{outletInlet@\string\OFboundary{outletInlet}!きょうかいじょうけん@境界条件}%
\index{きょうかいじょうけん@境界条件!outletInlet@\string\OFboundary{outletInlet}}%
 \OFboundary{outletInlet} &
     $\bm{U}$の向きによって\OFboundary{fixedValue}と
     \OFboundary{zeroGradient}の間で$\bm{U}$と$p$を切り替える &
         \OFkeyword{outletValue},\OFkeyword{value} \\
 \OFboundary{pressureInletOutletVelocity} &
     \OFboundary{pressureInletVelocity}と
     \OFboundary{inletOutlet}の組み合わせ & \OFkeyword{value} \\
 \OFboundary{pressureDirectedInletOutletVelocity} &
     \OFboundary{pressureDirectedInletVelocity}と\OFboundary{inletOutlet}の組み合わせ &
         \OFkeyword{value},\OFkeyword{inletDirection} \\
\index{pressureTransmissive@\string\OFboundary{pressureTransmissive}!きょうかいじょうけん@境界条件}%
\index{きょうかいじょうけん@境界条件!pressureTransmissive@\string\OFboundary{pressureTransmissive}}%
 \OFboundary{pressureTransmissive} &
     周囲の圧力$p_{\infty}$に超音速圧縮波を伝える & \OFkeyword{pInf} \\
\index{supersonicFreeStream@\string\OFboundary{supersonicFreeStream}!きょうかいじょうけん@境界条件}%
\index{きょうかいじょうけん@境界条件!supersonicFreeStream@\string\OFboundary{supersonicFreeStream}}%
 \OFboundary{supersonicFreeStream} &
     斜めの衝撃を$p_{\infty}$,$T_{\infty}$,$U_{\infty}$の環境に伝える &
         \OFkeyword{pInf},\OFkeyword{TInf},\OFkeyword{UInf} \\
 その他 \\
 \hline
 \tblstrut
\index{slip@\string\OFboundary{slip}!きょうかいじょうけん@境界条件}%
\index{きょうかいじょうけん@境界条件!slip@\string\OFboundary{slip}}%
 \OFboundary{slip} & $\phi$がスカラなら\OFboundary{zeroGradient},
     $\phi$がベクトルなら法線成分は\OFboundary{fixedValue 0}で,
     接線成分は\OFboundary{zeroGradient} & --- \\
\index{partialSlip@\string\OFboundary{partialSlip}!きょうかいじょうけん@境界条件}%
\index{きょうかいじょうけん@境界条件!partialSlip@\string\OFboundary{partialSlip}}%
 \OFboundary{partialSlip} &
     混合\OFboundary{zeroGradient}/\OFboundary{slip}条件は
     \OFkeyword{valueFraction}による.$0$ならば\OFboundary{slip}. &
         \OFkeyword{valueFraction} \\
 \hline
\end{tabularx}
