%#! platex UserGuideJa
\begin{tabularx}{\textwidth}{llX}
 キーワード & オプション & 説明 \\
 \hline
\index{interpolationScheme@\string\OFkeyword{interpolationScheme}!キーワード}%
\index{キーワード!interpolationScheme@\string\OFkeyword{interpolationScheme}}%
 \OFkeyword{interpolationScheme} &
\index{cell@\string\OFkeyword{cell}!キーワードエントリ}%
\index{キーワードエントリ!cell@\string\OFkeyword{cell}}%
     \OFkeyword{cell} &
         セル中心の値でセル全体が一定とみなす \\
 &
\index{cellPoint@\string\OFkeyword{cellPoint}!キーワードエントリ}%
\index{キーワードエントリ!cellPoint@\string\OFkeyword{cellPoint}}%
     \OFkeyword{cellPoint} &
         セルの値から線型重み付け補間 \\
 &
\index{cellPointFace@\string\OFkeyword{cellPointFace}!キーワードエントリ}%
\index{キーワードエントリ!cellPointFace@\string\OFkeyword{cellPointFace}}%
     \OFkeyword{cellPointFace} &
         線型重み付けまたはセル界面から補間 \\
\index{setFormat@\string\OFkeyword{setFormat}!キーワード}%
\index{キーワード!setFormat@\string\OFkeyword{setFormat}}%
 \OFkeyword{setFormat} &
\index{raw@\string\OFkeyword{raw}!キーワードエントリ}%
\index{キーワードエントリ!raw@\string\OFkeyword{raw}}%
     \OFkeyword{raw} &
         ASCII生データ \\
 &
\index{gnuplot@\string\OFkeyword{gnuplot}!キーワードエントリ}%
\index{キーワードエントリ!gnuplot@\string\OFkeyword{gnuplot}}%
     \OFkeyword{gnuplot} &
         gnuplot形式データ \\
 &
\index{xmgr@\string\OFkeyword{xmgr}!キーワードエントリ}%
\index{キーワードエントリ!xmgr@\string\OFkeyword{xmgr}}%
     \OFkeyword{xmgr} &
         Grace/xmgr形式データ \\
 &
\index{jplot@\string\OFkeyword{jplot}!キーワードエントリ}%
\index{キーワードエントリ!jplot@\string\OFkeyword{jplot}}%
     \OFkeyword{jplot} &
         jPlot形式データ \\
\index{surfaceFormat@\string\OFkeyword{surfaceFormat}!キーワード}%
\index{キーワード!surfaceFormat@\string\OFkeyword{surfaceFormat}}%
 \OFkeyword{surfaceFormat} &
\index{null@\string\OFkeyword{null}!キーワードエントリ}%
\index{キーワードエントリ!null@\string\OFkeyword{null}}%
     \OFkeyword{null} &
         出力しない \\
 &
\index{foamFile@\string\OFkeyword{foamFile}!キーワードエントリ}%
\index{キーワードエントリ!foamFile@\string\OFkeyword{foamFile}}%
     \OFkeyword{foamFile} &
         点,面,値のファイル \\
 &
\index{dx@\string\OFkeyword{dx}!キーワードエントリ}%
\index{キーワードエントリ!dx@\string\OFkeyword{dx}}%
     \OFkeyword{dx} &
         DXスカラまたはベクトル形式 \\
 &
\index{vtk@\string\OFkeyword{vtk}!キーワードエントリ}%
\index{キーワードエントリ!vtk@\string\OFkeyword{vtk}}%
     \OFkeyword{vtk} &
         VTK ASCII形式 \\
 &
\index{raw@\string\OFkeyword{raw}!キーワードエントリ}%
\index{キーワードエントリ!raw@\string\OFkeyword{raw}}%
     \OFkeyword{raw} &
         $xyz$座標と値.gnuplotの\texttt{splot}などで使われる \\
 &
\index{stl@\string\OFkeyword{stl}!キーワードエントリ}%
\index{キーワードエントリ!stl@\string\OFkeyword{stl}}%
     \OFkeyword{stl} &
         ASCII STL.表面のみ,値なし \\
 \\
\index{fields@\string\OFkeyword{fields}!キーワード}%
\index{キーワード!fields@\string\OFkeyword{fields}}%
 \OFkeyword{fields} &
     \multicolumn{2}{l}{サンプルするフィールドのリスト,
     たとえば速度\OFkeyword{U}の場合,} \\
 &
     \OFkeyword{U} &
         $\bm{U}$の全成分を出力 \\
 \\
\index{sets@\string\OFkeyword{sets}!キーワード}%
\index{キーワード!sets@\string\OFkeyword{sets}}%
 \OFkeyword{sets} &
     \multicolumn{2}{l}{1次元のsetsサブディクショナリのリスト.\autoref{tbl:6.4}を参照} \\
\index{surfaces@\string\OFkeyword{surfaces}!キーワード}%
\index{キーワード!surfaces@\string\OFkeyword{surfaces}}%
 \OFkeyword{surfaces} &
     \multicolumn{2}{l}{2次元のsurfacesサブディクショナリリスト.
     \autoref{tbl:6.5}および\autoref{tbl:6.6}を参照} \\
 \hline
\end{tabularx}
