%#! platex UserGuideJa
\begin{tabularx}{\textwidth}{Xll}
 キーワード & オプション & 説明 \\
 \hline
\index{interpolationScheme@\OFkeyword{interpolationScheme}!キーワード}%
\index{キーワード!interpolationScheme@\OFkeyword{interpolationScheme}}%
 \OFkeyword{interpolationScheme} &
\index{cell@\OFkeyword{cell}!キーワードエントリ}%
\index{キーワードエントリ!cell@\OFkeyword{cell}}%
     \OFkeyword{cell} &
         セル中心の値でセル全体が一定とみなす \\
 &
\index{cellPoint@\OFkeyword{cellPoint}!キーワードエントリ}%
\index{キーワードエントリ!cellPoint@\OFkeyword{cellPoint}}%
     \OFkeyword{cellPoint} &
         セルの値から線型重み付け補間 \\
 &
\index{cellPointFace@\OFkeyword{cellPointFace}!キーワードエントリ}%
\index{キーワードエントリ!cellPointFace@\OFkeyword{cellPointFace}}%
     \OFkeyword{cellPointFace} &
         線型重み付けまたはセル表面から補間 \\
\index{setFormat@\OFkeyword{setFormat}!キーワード}%
\index{キーワード!setFormat@\OFkeyword{setFormat}}%
 \OFkeyword{setFormat} &
\index{raw@\OFkeyword{raw}!キーワードエントリ}%
\index{キーワードエントリ!raw@\OFkeyword{raw}}%
     \OFkeyword{raw} &
         ASCII生データ \\
 &
\index{gnuplot@\OFkeyword{gnuplot}!キーワードエントリ}%
\index{キーワードエントリ!gnuplot@\OFkeyword{gnuplot}}%
     \OFkeyword{gnuplot} &
         gnuplot形式データ \\
 &
\index{xmgr@\OFkeyword{xmgr}!キーワードエントリ}%
\index{キーワードエントリ!xmgr@\OFkeyword{xmgr}}%
     \OFkeyword{xmgr} &
         Grace/xmgr形式データ \\
 &
\index{jplot@\OFkeyword{jplot}!キーワードエントリ}%
\index{キーワードエントリ!jplot@\OFkeyword{jplot}}%
     \OFkeyword{jplot} &
         jPlot形式データ \\
\index{surfaceFormat@\OFkeyword{surfaceFormat}!キーワード}%
\index{キーワード!surfaceFormat@\OFkeyword{surfaceFormat}}%
 \OFkeyword{surfaceFormat} &
\index{null@\OFkeyword{null}!キーワードエントリ}%
\index{キーワードエントリ!null@\OFkeyword{null}}%
     \OFkeyword{null} &
         出力しない \\
 &
\index{foamFile@\OFkeyword{foamFile}!キーワードエントリ}%
\index{キーワードエントリ!foamFile@\OFkeyword{foamFile}}%
     \OFkeyword{foamFile} &
         点,面,値のファイル \\
 &
\index{dx@\OFkeyword{dx}!キーワードエントリ}%
\index{キーワードエントリ!dx@\OFkeyword{dx}}%
     \OFkeyword{dx} &
         DXスカラまたはベクトル形式 \\
 &
\index{vtk@\OFkeyword{vtk}!キーワードエントリ}%
\index{キーワードエントリ!vtk@\OFkeyword{vtk}}%
     \OFkeyword{vtk} &
         VTK ASCII形式 \\
 &
\index{raw@\OFkeyword{raw}!キーワードエントリ}%
\index{キーワードエントリ!raw@\OFkeyword{raw}}%
     \OFkeyword{raw} &
         $xyz$座標と値.gnuplotの\texttt{splot}などで使われる \\
 &
\index{stl@\OFkeyword{stl}!キーワードエントリ}%
\index{キーワードエントリ!stl@\OFkeyword{stl}}%
     \OFkeyword{stl} &
         ASCII STL.表面のみ,値なし \\
\index{fields@\OFkeyword{fields}!キーワード}%
\index{キーワード!fields@\OFkeyword{fields}}%
 \OFkeyword{fields} &
     \multicolumn{2}{l}{サンプルするフィールドのリスト,
     たとえば速度\OFkeyword{U}の場合,} \\
 &
     \OFkeyword{U} &
         $\bm{U}$の全成分を出力 \\
 &
     \OFkeyword{U.component(0)} &
         成分0を出力.つまり$U_{x}$ \\
 &
     \OFkeyword{U.component(1)} &
         成分1を出力.つまり$U_{y}$ \\
 &
     \OFkeyword{mag(U)} &
         ベクトル,テンソルの大きさを出力.つまり$|\bm{U}|$ \\
\index{sets@\OFkeyword{sets}!キーワード}%
\index{キーワード!sets@\OFkeyword{sets}}%
 \OFkeyword{sets} &
     \multicolumn{2}{l}{1次元のsetsサブディクショナリのリスト.\autoref{tbl:6.4}を参照} \\
\index{surfaces@\OFkeyword{surfaces}!キーワード}%
\index{キーワード!surfaces@\OFkeyword{surfaces}}%
 \OFkeyword{surfaces} &
     \multicolumn{2}{l}{2次元のsurfacesサブディクショナリリスト.
     \autoref{tbl:6.5}および\autoref{tbl:6.6}を参照} \\
 \hline
\end{tabularx}
